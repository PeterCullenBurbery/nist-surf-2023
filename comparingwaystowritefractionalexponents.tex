\documentclass{article}
\usepackage{amsmath}% For the equation* environment
\usepackage[euler-digits,euler-hat-accent]{eulervm}
\usepackage{newpxtext}
\begin{document}
\section{First example}

The well-known Pythagorean theorem \(x^2 + y^2 = z^2\) was proved to be invalid for other exponents, meaning the next equation has no integer solutions for \(n>2\):

\[ x^n + y^n = z^n \]

\section{Second example}

This is a simple math expression \(\sqrt{x^2+1}\) inside text. 
And this is also the same: 
\begin{math}
\sqrt{x^2+1}
\end{math}
but by using another command.

This is a simple math expression without numbering
\[\sqrt{x^2+1}\] 
separated from text.

This is also the same:
\begin{displaymath}
\sqrt{x^2+1}
\end{displaymath}

\ldots and this:
\begin{equation*}
\sqrt{x^2+1}
\end{equation*}
\section{Different ways of writing fractional exponents}
\begin{flalign}
q^{1/2}
\end{flalign}
\begin{flalign}
q^{\frac{1}{2}}
\end{flalign}
It seems that the way of writing the exponent with frac takes up less horizontal space than writing the fraction on one line without frac.
Here is a more complicated example
\begin{flalign}
    q^{1/4\left(2j+1\right)}
\end{flalign}
\begin{flalign}
    q^{\frac{1}{4}\left(2j+1\right)}
\end{flalign}

\begin{flalign}
    q^{\frac{\left(2j+1\right)}{4}}
\end{flalign}
It seems the most compact way is the third way.
\subsection{coefficient in numerator not 1}
Here is another example where there is a coefficient in the numerator that is not 1.
\begin{flalign}
    q^{3/4\left(2j+1\right)}
\end{flalign}
\begin{flalign}
    q^{\frac{3}{4}\left(2j+1\right)}
\end{flalign}

\begin{flalign}
    q^{\frac{3\left(2j+1\right)}{4}}
\end{flalign}
Again, it seems the most compact representation is the third one.
\end{document}