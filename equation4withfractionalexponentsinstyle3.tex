\documentclass[fleqn]{article}
\usepackage{graphicx} % Required for inserting images
\usepackage[a3paper]{geometry}
\usepackage{color}
\usepackage{amsmath}
\usepackage{ amssymb}
\counterwithin*{equation}{section}
\title{manual translation}
\author{Peter Burbery}
\date{June 2023}
\usepackage[euler-digits,euler-hat-accent]{eulervm}
\newcommand{\Whyp}[5]{\,\mbox{}_{#1}W_{#2}\!\left({#3};{#4};{#5}\right)}
\usepackage{newpxtext}
\newcommand{\qhyp}[5]{\,\mbox{}_{#1}\phi_{#2}\!\left(
\begin{array}{c}{#3}\\[0.10cm]{#4}\end{array};{#5}\right)}
\newcommand{\qphyp}[6]{\,{}_{#1}\phi_{{#2}}^{{#3}}\!\left(
\begin{array}{c}{#4}\\[0.10cm] {#5}\end{array};#6\right)}
\newcommand{\qPhihyp}[5]{\Phi{}_{#1}^{#2}\!\left[
\begin{array}{c}{#3}\\ {#4}\end{array};#5\right]}
\newcommand{\Kampe}[5]{F_{#1}^{#2}\!\left[
\begin{array}{c}{#3}\\ {#4}\end{array};#5\right]}
\newcommand{\hyp}[5]{\,\mbox{}_{#1}F_{#2}\!\left(
\begin{array}{c}{#3}\\ {#4}\end{array};#5\right)}


\newcommand\moro[1]{{\textcolor{blue}{#1}}}
\newcommand\moqo[1]{{\textcolor{green!55!black}{#1}}}
\newcommand\soro[1]{{\textcolor{red!40!blue}{\sout #1}}}
\newcommand\noro[1]{{\textcolor{red}{#1}}}
\newcommand\poro[1]{{\textcolor{purple}{#1}}}
\newcommand\doro[1]{{\textcolor{darkgray}{#1}}}
\newcommand\boro[1]{{\textcolor{black}{#1}}}

\newcommand{\dsA}{\ensuremath{\mathbb{A}}}
\newcommand{\dsB}{\ensuremath{\mathbb{B}}}
\newcommand{\dsC}{\ensuremath{\mathbb{C}}}
\newcommand{\dsD}{\ensuremath{\mathbb{D}}}
\newcommand{\dsE}{\ensuremath{\mathbb{E}}}
\newcommand{\dsF}{\ensuremath{\mathbb{F}}}
\newcommand{\dsG}{\ensuremath{\mathbb{G}}}
\newcommand{\dsH}{\ensuremath{\mathbb{H}}}
\newcommand{\dsI}{\ensuremath{\mathbb{I}}}
\newcommand{\dsJ}{\ensuremath{\mathbb{J}}}
\newcommand{\dsK}{\ensuremath{\mathbb{K}}}
\newcommand{\dsL}{\ensuremath{\mathbb{L}}}
\newcommand{\dsM}{\ensuremath{\mathbb{M}}}
\newcommand{\dsN}{\ensuremath{\mathbb{N}}}
\newcommand{\dsO}{\ensuremath{\mathbb{O}}}
\newcommand{\dsP}{\ensuremath{\mathbb{P}}}
\newcommand{\dsQ}{\ensuremath{\mathbb{Q}}}
\newcommand{\dsR}{\ensuremath{\mathbb{R}}}
\newcommand{\dsS}{\ensuremath{\mathbb{S}}}
\newcommand{\dsT}{\ensuremath{\mathbb{T}}}
\newcommand{\dsU}{\ensuremath{\mathbb{U}}}
\newcommand{\dsV}{\ensuremath{\mathbb{V}}}
\newcommand{\dsW}{\ensuremath{\mathbb{W}}}
\newcommand{\dsX}{\ensuremath{\mathbb{X}}}
\newcommand{\dsY}{\ensuremath{\mathbb{Y}}}
\newcommand{\dsZ}{\ensuremath{\mathbb{Z}}}

\newcommand{\scA}{\ensuremath{\mathcal{A}}}
\newcommand{\scB}{\ensuremath{\mathcal{B}}}
\newcommand{\scC}{\ensuremath{\mathcal{C}}}
\newcommand{\scD}{\ensuremath{\mathcal{D}}}
\newcommand{\scE}{\ensuremath{\mathcal{E}}}
\newcommand{\scF}{\ensuremath{\mathcal{F}}}
\newcommand{\scG}{\ensuremath{\mathcal{G}}}
\newcommand{\scH}{\ensuremath{\mathcal{H}}}
\newcommand{\scI}{\ensuremath{\mathcal{I}}}
\newcommand{\scJ}{\ensuremath{\mathcal{J}}}
\newcommand{\scK}{\ensuremath{\mathcal{K}}}
\newcommand{\scL}{\ensuremath{\mathcal{L}}}
\newcommand{\scM}{\ensuremath{\mathcal{M}}}
\newcommand{\scN}{\ensuremath{\mathcal{N}}}
\newcommand{\scO}{\ensuremath{\mathcal{O}}}
\newcommand{\scP}{\ensuremath{\mathcal{P}}}
\newcommand{\scQ}{\ensuremath{\mathcal{Q}}}
\newcommand{\scR}{\ensuremath{\mathcal{R}}}
\newcommand{\scS}{\ensuremath{\mathcal{S}}}
\newcommand{\scT}{\ensuremath{\mathcal{T}}}
\newcommand{\scU}{\ensuremath{\mathcal{U}}}
\newcommand{\scV}{\ensuremath{\mathcal{V}}}
\newcommand{\scW}{\ensuremath{\mathcal{W}}}
\newcommand{\scX}{\ensuremath{\mathcal{X}}}
\newcommand{\scY}{\ensuremath{\mathcal{Y}}}
\newcommand{\scZ}{\ensuremath{\mathcal{Z}}}

\begin{document}

\maketitle

\section{Introduction}

\section{equation 5}
\begin{flalign}
&\Whyp{12}{11}{a}{b,c,d,e,f,g,h,i,j}{q,z}=\qhyp{12}{11}{a,\pm q\sqrt{a},b,c,d,e,f,g,h,i,j}{\pm \sqrt{a},q \frac{a}{b},q \frac{a}{c},q \frac{a}{d},q \frac{a}{e},q \frac{a}{f},q \frac{a}{g},q \frac{a}{h},q \frac{a}{i},q \frac{a}{j}}{q,z}
&&
\end{flalign}

\section{equation 4}

\begin{flalign}
    \Whyp{8}{7}{a}{b,c,d,e,f}{q,z}=\sum_{n=0}^{\infty}
\nonumber && \\
&\times    
  \sum_{m=0}^{\infty}
\nonumber && \\
&\times
  \sum_{j=0}^{\infty}
\nonumber && \\
&\times
\int_{-\pi}^{\pi}
\frac{\left(
q^{\frac{2j+2m+2n+5}{4}}\frac{ a^{5/4} \sqrt{f} \rho }{g \sqrt{b c d e z}}, 
q^{\frac{-\left(2j+2m+2n+1\right)}{4}} \frac{\rho \sqrt{b c d e z}}{a^{5/4} \sqrt{f} g}
, q^{\frac{2j+2m+2n+5}{4}} \frac{a^{\frac{5}{4}} \sqrt{f } g }{\rho \sqrt{b c d e z}}
;q
\right)_{\infty}}{\left( \right)_{\infty}}
\,{\mathrm d}\zeta
  && 
\end{flalign}

\section{simplified form of equation 4}

\begin{flalign}
\dsW=\frac{\dsU}{2\pi\dsV} \sum_{n=0}^{\infty}
{\dsP^n \frac{\dsQ \dsR}{\dsS \dsT}}
\sum_{m=0}^{\infty} \dsA^m \frac{\dsB \dsC \dsD}{\dsE \dsF \dsG} \sum_{j=0}^{\infty}q^\dsH \frac{ \dsI^j \dsJ \dsK}{\dsL \dsM} \int_{-\pi}^{\pi} \frac{\dsN}{\dsO} \,{\mathrm d}\zeta
\end{flalign}

\subsection{left-hand side of the equation defined with \dsW}
This is what we will be defining:
\begin{flalign}
    \dsW
\end{flalign}

\begin{flalign}
    \dsW=\Whyp{8}{7}{a}{b,c,d,e,f}{q,z}
\end{flalign}

\subsection{stuff in front of the sum with index variable n defined in terms of \dsU  and \dsV}
This is what we will be defining.
\begin{flalign}
    \frac{\dsU}{2 \pi\dsV}
\end{flalign}

\begin{flalign}
    \dsU=\left( q, \dsX ;q\right)_{\infty}
\end{flalign}

\begin{flalign}
    \dsX=q^{-4} \left(\frac{b c d e f z}{a^2}\right)^2
\end{flalign}

\begin{flalign}
    \dsV=\left( f,\frac{q}{f} ,\dsY;q \right)_{\infty}
\end{flalign}
\begin{flalign}
    \dsY= q^{-3} \left(\frac{b c d e f z}{a^2}\right)^2
\end{flalign}


\subsection{first sum with index variable n defined with \dsP, \dsQ, \dsR, \dsS, and \dsT}

This is what we will be defining:
\begin{flalign}
     \sum_{n=0}^{\infty}
{\dsP^n \frac{\dsQ \dsR}{\dsS \dsT}}
\end{flalign}
\begin{equation} \label{eq1}
\begin{split}
\dsP & = \left( \frac{b c z}{\sqrt{q a}}\right) \\
 & = \left( \frac{b c z}{q^{\frac{1}{2}} a^{\frac{1}{2}} }
\right) \\ \nonumber
& = q^{\frac{-1}{2}} 
 \left( \frac{b c z}{a^{\frac{1}{2}} }
\right)
\end{split}
\end{equation}

\begin{equation} \label{eq1}
\begin{split}
\dsP^n & = \left( q^{\frac{-1}{2}} 
 \left( \frac{b c z}{a^{\frac{1}{2}} }
\right)\right)^n \\
 & = \left(q^{\frac{-1}{2}} \right)^n \left( \frac{b c z}{a^{\frac{1}{2}} }
\right)^n \\ \nonumber
& = q^{\frac{-1}{2} \times n}  \left( b c z\right) ^n \left(a^{\frac{-1}{2}} 
\right)^n \\
& = q^{\frac{-n}{2}} \left( b c z\right)^n \left(a^{\frac{-1}{2}\times n}
\right) \\
& = q^{\frac{-n}{2}} \left( b c z\right)^n a^{\frac{-n}{2}} \\
& =
q^{\frac{-n}{2}} b^n c^n z^n a^{\frac{-n}{2}}
\end{split}
\end{equation}
I think I prefer the form \( q^{\frac{-n}{2}} \left( b c z\right)^n a^{\frac{-n}{2}}\) for \(\dsP^n\).

\begin{flalign}
    \dsQ=\left(q a;q\right)_{2n}
\end{flalign}

\begin{flalign}
    \dsR=\left(\dsZ;\scA ,\scB, \scC,d,e,f; q\right)_n
\end{flalign}
\begin{flalign}
    \dsZ=q^{\frac{1}{2}}\frac{a^{\frac{3}{2}}}{b c}=\sqrt{q} \frac{a^{\frac{3}{2}}}{b c}
\end{flalign}

\begin{flalign}
    \scA=\sqrt{q} \frac{a}{b}
\end{flalign}

\begin{flalign}
    \scB=\sqrt{q} \frac{a}{c}
\end{flalign}

\begin{flalign}
    \scC=q \frac{a}{b c}
\end{flalign}

\begin{flalign}
    \dsS=\left(\scD;q\right)_{2n}
\end{flalign}

\begin{flalign}
    \scD=\sqrt{q}\frac{a^{\frac{3}{2}}}{b c}
\end{flalign}

\begin{flalign}
    \dsT=\left(q, \sqrt{qa}, \scE,\scF,\scG,\scH, \scI ;q\right)_n
\end{flalign}

\begin{flalign}
    \scE=q \frac{a}{b}
\end{flalign}


\begin{flalign}
    \scF=q \frac{a}{c}
\end{flalign}

\begin{flalign}
    \scG=q \frac{a}{d}
\end{flalign}

\begin{flalign}
    \scH=q \frac{a}{e}
\end{flalign}


\begin{flalign}
    \scI=q \frac{a}{f}
\end{flalign}

\subsection{second sum with the index variable m defined in terms of \dsA, \dsB, \dsC, \dsD, \dsE, \dsF and \dsG}

This is what we will be defining.
\begin{flalign}
    \sum_{m=0}^{\infty} \dsA^m \frac{\dsB \dsC \dsD}{\dsE \dsF \dsG}
\end{flalign}

\begin{flalign}
    \dsA=\left(\frac{b c d z}{q a}\right)
\end{flalign}

\begin{flalign}
    \dsB=\left( q^{2n+1} a;q\right)_{2m}
\end{flalign}

\begin{flalign}
    \dsC=\left(\scJ, \scK, \scL, \scM, q^n e,q^n f ;q \right)_m
\end{flalign}

\begin{flalign}
    \scJ= q^{2n+1} \frac{a^2}{b c d}
\end{flalign}

\begin{flalign}
    \scK=q^{n+1}\frac{a}{bc}
\end{flalign}

\begin{flalign}
    \scL=\sqrt{q}\frac{\sqrt{a}}{b}
\end{flalign}


\begin{flalign} 
\begin{split}
\scM & = q^n \frac{\left( q a\right)^{\frac{3}{2}}}{b c d} \\
 & = q^n \frac{q^{\frac{3}{2}} a^{\frac{3}{2}}}{b c d} \\
  & = q^n q^{\frac{3}{2}} \frac{ a^{\frac{3}{2}}}{b c d} \\
    & =  q^{\frac{2n+3}{2}} \frac{ a^{\frac{3}{2}}}{b c d} 
\end{split}
\end{flalign}

\begin{flalign}
    \dsD=\left( \scN , \scO ;q \right)_{\infty}
\end{flalign}

\begin{flalign}
    \scN=q^{2\left(m+n\right)+1} a
\end{flalign}


\begin{flalign} 
\begin{split}
  \scO & = q^{m+n} \frac{\left(q a\right)^{\frac{5}{2}}}{b c d e f} \\
 & = q^{m+n} \frac{q^{\frac{5}{2}} a^{\frac{5}{2}}}{b c d e f} \\
  & = q^{m+n} q^{\frac{5}{2}} \frac{ a^{\frac{3}{2}}}{b c d e f} \\
    & =  q^{\frac{2m+2n+5}{2}} \frac{ a^{\frac{3}{2}}}{b c d e f} 
\end{split}
\end{flalign}

\begin{flalign}
    \dsE=\left( \frac{q^{2n+1} a^2}{b c d} ;q\right)_{2m}
\end{flalign}

\begin{flalign}
    \dsF=\left( q,\scP,\scQ,\scS,\scT,\scU;q\right)_m
\end{flalign}



\begin{flalign} 
\begin{split}
  \scP & = q^n \sqrt{q a} \\
 & = q^n \left( q a\right)^{\frac{1}{2}} \\
  & = q^n q^{\frac{1}{2}} a^{\frac{1}{2}} \\
    & =  q^{n+\frac{1}{2}} a^{\frac{1}{2}} \\
     & =  q^{\frac{2n+1}{2}} a^{\frac{1}{2}} 
\end{split}
\end{flalign}

\begin{flalign}
    \scQ=\frac{\scR}{b c}
\end{flalign}

\begin{flalign} 
\begin{split}
  \scR & = q^{2n} \left(q a\right)^{\frac{3}{2}} \\
  & = q^{2n} q^{\frac{3}{2}} a^{\frac{3}{2}} \\
    & =  q^{2n+\frac{3}{2}} a^{\frac{3}{2}} \\
     & =  q^{\frac{4n+3}{2}} a^{\frac{3}{2}} 
\end{split}
\end{flalign}


\begin{flalign}
    \scS=q^{n+1} \frac{a}{d}
\end{flalign}

\begin{flalign}
    \scT=q^{n+1} \frac{a}{e}
\end{flalign}

\begin{flalign}
    \scU=q^{n+1} \frac{a}{f}
\end{flalign}



\begin{flalign}
    \dsG=\left(\scV,\scW,\scX,\scY;q\right)_{\infty}
\end{flalign}


\begin{flalign} 
\begin{split}
  \scV & = q^{m+n} \frac{\left(q a\right)^{\frac{5}{2}}}{b c d e f} \\
 & = q^{m+n} \frac{q^{\frac{5}{2}} a^{\frac{5}{2}}}{b c d e f} \\
  & = q^{m+n} q^{\frac{5}{2}} \frac{ a^{\frac{3}{2}}}{b c d e f} \\
    & =  q^{\frac{2m+2n+5}{2}} \frac{ a^{\frac{3}{2}}}{b c d e f} 
\end{split}
\end{flalign}

\begin{flalign}
    \scW=\frac{b c d e z}{\scZ}
\end{flalign}

\begin{flalign} 
\begin{split}
  \scZ & = q^{m+n+\frac{3}{2}}a^{\frac{5}{2}} \\
    & =  q^{\frac{2m+2n+3}{2}} a^{\frac{5}{2}}
\end{split}
\end{flalign}

\begin{flalign}
    \scX=q^{m+n+1} \frac{a}{f}
\end{flalign}

\begin{flalign}
    \scY=
\end{flalign}





\end{document}




